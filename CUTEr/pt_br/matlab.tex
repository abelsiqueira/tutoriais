\subsection{Exemplos em MATLAB}
Com o {\tt CUTEr} devidamente instalado, vejamos como ocorre a interface com o {\tt MATLAB}.

Primeiro, crie uma pasta {\tt Test}, onde ser\~ao gerados os arquivos decodificados do problema
 e o arquivo {\tt .mex} da interface para {\tt MATLAB}.

No terminal, dentro da pasta {\tt Test}, execute o comando:
\begin{terminal}
$ runcuter -p mx -D  ROSENBR
\end{terminal}
Al\'em dos arquivos padr\~ao, gerados pelo decoder para o problema {\tt ROSENBR}, haver\'a 
tamb\'em o arquivo: {\tt mcuter.mex}. A extens\~ao {\tt .mex} poder\'a variar conforme a 
instala\c{c}\~ao do {\tt CUTEr} e o sistema operacional.

Em seguida, v\'a para o {\tt MATLAB}. Adicione as pastas {\tt \$CUTER/common/src/matlab} e {\tt 
\$MYCUTER/bin} ao {\tt path} do {\tt MATLAB}. Isso pode ser feito atrav\'es do menu {\tt File\ 
>\ Set Path...} ou pelo prompt do {\tt MATLAB} usando o comando {\tt addpath}.

Dentro do {\tt MATLAB}, v\'a para a pasta {\tt Test}. Para verificar se est\'a tudo certo, 
execute o comando:
\begin{code}{matlab}
>> prob = cuter_setup()

prob = 

         n: 2
         m: 0
      nnzh: 3
      nnzj: 0
         x: [2x1 double]
        bl: [2x1 double]
        bu: [2x1 double]
         v: [0x1 double]
        cl: [0x1 double]
        cu: [0x1 double]
    equatn: [0x1 logical]
    linear: [0x1 logical]
      name: 'ROSENBR   '
\end{code}
Se estiver tudo correto, a sa\'ida dever\'a ser como acima.

O comando {\tt cuter\_setup}, que inicializa o problema e fornece suas carcacter\'isticas, se 
encontra na pasta {\tt \$CUTER/src/common/matlab}, dentro da qual est\~ao as demais rotinas 
(arquivos {\tt .m}) para acessar fun\c{c}\~ao objetivo, gradiente, restri\c{c}\~oes, etc.

Por exemplo, para saber o valor da fun\c{c}\~ao objetivo em um determinado ponto, usamos:
\begin{code}{matlab}
>> f = cuter_obj([-1 2]')

f =

   104
\end{code}
Abaixo temos um c\'odigo em {\tt MATLAB} que implementa o m\'etodo do gradiente. Para acessar a 
fun\c{c}\~ao objetivo e o gradiente, fizemos uso da fun\c{c}\~ao {\tt cuter\_obj}.
\begin{code}{matlab}
function [x,f,gradnorm,iter,flag] = cute_gradient()
%
%   [x,f,gradnorm,iter,flag] = cute_gradient()
%
%   flags:
%   -2      maximum number of iterations reached
%   -1      too short step size
%    1      gradient norm less than eps0
%

    % inicializando o problema
    prob = cuter_setup();
    
    % dimensao do problema
    n = prob.n;
    
    % ponto inicial
    x0 = prob.x;
    
    %====================================
    % Gradient method with linesearch
    %====================================
    maxit=n*10000;
    gamma=1e-4;
    eps0=1e-5;
    tmin=1e-8;
    done=0;
    flag=0;
    k=0;
    x=x0;
    
    while ~done
        k=k+1;
        
        % maximum number of iterations test
        if k>maxit
           done=1;
           flag=-2;
           continue
        end
        
        x0=x;
        
        [f,g] = cuter_obj(x);
        f0=f;
        
        gradnorm=norm(g,2);
        
        % gradient norm test
        if gradnorm<eps0
            done=1;
            flag=1;
            continue
        end
        
        % search direction
        d = -g;
        gtd = g'*d;
        t = 1;
        
        x = x0 + t*d;
        f = cuter_obj(x);
        
        % Armijo linesearch
        while f > f0 + t*gamma*gtd
            t = 0.5*t;
            if t<tmin
                done=1;
                flag=-1;
            end
            
            x = x0 + t*d;
            f = cuter_obj(x);
        end
        
    end
    
    iter=k;

end
\end{code}
E executando o c\'odigo acima, obtemos
\begin{code}{matlab}
>> [x,f,gradnorm,iter,flag] = cute_gradient()

x =

    1.0000
    1.0000


f =

   6.1319e-11


gradnorm =

   9.9971e-06


iter =

       10917


flag =

     1
\end{code}

\'E poss\'ivel tamb\'em utilizar as rotinas de otimiza\c{c}\~ao do pr\'oprio {\tt MATLAB}. Por 
exemplo, para utilizar o {\tt fminunc}, usamos:
\begin{code}{matlab}
>> prob = cuter_setup();
>> [x,f,flag] = fminunc(@(x) cuter_obj(x),prob.x)
Warning: Gradient must be provided for trust-region method;
  using line-search method instead. 
> In fminunc at 356

Local minimum found.

Optimization completed because the size of the gradient is less than
the default value of the function tolerance.

<stopping criteria details>


x =

    1.0000
    1.0000


f =

   2.8336e-11


flag =

     1
\end{code}
Lembrando que as op\c{c}\~oes do solver s\~ao ajustadas pelo comando {\tt optimset}.
\begin{code}{matlab}
>> prob = cuter_setup();
>> opts = optimset('GradObj','on');
>> [x,f,flag] = fminunc(@(x) cuter_obj(x),prob.x,opts)

Local minimum possible.

fminunc stopped because the final change in function value relative to 
its initial value is less than the default value of the function tolerance.

<stopping criteria details>


x =

    1.0000
    1.0000


f =

   4.0035e-13


flag =

     3
\end{code}
