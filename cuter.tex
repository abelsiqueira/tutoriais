\documentclass[letterpaper,11pt]{article}
\usepackage[brazil]{babel}
\usepackage[T1]{fontenc}
\usepackage{ae}
\usepackage[utf8]{inputenc}
\usepackage[dvipsnames]{color}
\usepackage{graphicx}
\usepackage{epsfig}
\usepackage{multirow}
\usepackage{amssymb, amsmath, amsfonts}
\bibliographystyle{plain}
\topmargin		0 cm
\hoffset		0 cm
\voffset		0 cm
\evensidemargin		0 cm
\oddsidemargin		0 cm
\setlength{\textwidth}{16 cm}
\setlength{\textheight}{21 cm}
\usepackage{pnl}
\defineshdwidth{0.1 cm}

\title{Apostila de Instalação do CUTEr}
\author{Abel Soares Siqueira}

\numberwithin{equation}{section}

% \newcommand{\visiblelbl}[1]{\label{#1}{\color{red}{#1}}}
\newcommand{\visiblelbl}[1]{\label{#1}}

\begin{document}
\maketitle

A biblioteca CUTEr (\cite{bib:cuter1,bib:cuter2}) é de extrema importância para a
realização de testes computacionais na área de otimização computacional. 
Tendo em vista que é necessário fazer comparações entre os algoritmos,
Muitos deles atualmente têm uma interface que permite a execução dos testes
do CUTEr. Vamos mostrar nesse tutorial como instalar o CUTEr e criar uma interface
em Fortran, C e C++ para o mesmo. Algumas partes deste tutorial supõem que o leitor
saiba o suficiente de linux.

\section{Instalação}

O CUTEr é um ambiente de testes para métodos de otimização. 
Ele foi feito em Fortran, porém tem suporte a C/C++.
Para instalar o CUTEr você irá precisar dos programas:
\begin{itemize}
 \item gawk 
 \item gcc 
 \item gfortran (ou outro compilador de fortran)
 \item svn (subversion)
\end{itemize}
Com os pacotes instalados, iremos criar uma pasta para o CUTEr. Você pode criar uma
pasta separada no sistema ou instalar tudo no seu diretório principal. No meu caso,
irei criar uma pasta chamada \verb+libraries+ na minha pasta pessoal, e criar uma
pasta para o CUTEr dentro dessa pasta. Ajuste os comandos de acordo com sua escolha.
\begin{verbatim}
$ mkdir -p $HOME/libraries/CUTEr
$ cd $HOME/libraries/CUTEr
\end{verbatim}
Agora faça o download do SifDec2 e do CUTEr2 pelo svn, usando os comandos{\color{red} *}
\small
\begin{verbatim}
$ svn co https://magi-trac-svn.mathappl.polymtl.ca/SVN/cuter/sifdec/branches/SifDec2 ./sifdec2 
$ svn co https://magi-trac-svn.mathappl.polymtl.ca/SVN/cuter/cuter/branches/CUTEr64 ./cuter2
\end{verbatim}
{\color{red}*Obs: Na última vez que tentei, o site estava fora do ar. Subi no site do LPOO
uma cópia da minha última versão, em 64 bits, de alguns meses. Os comandos para baixar e
 descompactar
essa versão são:
\begin{verbatim}
$ wget www.lpoo.ime.unicamp.br/CUTEr/sifdec2.tar.gz
$ wget www.lpoo.ime.unicamp.br/CUTEr/cuter2.tar.gz
$ tar -zxf sifdec2.tar.gz
$ tar -zxf cuter2.tar.gz
\end{verbatim}
Se você precisar da versão em 32 bits, rode o seguinte comando na pasta atualmente
\begin{verbatim}
$ sed -i 's/-m64//g' cuter2/config/{mac,all,linux}.cf
\end{verbatim}
}
\normalsize
Faça o download da coleção de problemas SIF na página
\begin{verbatim}
http://cuter.rl.ac.uk/cuter-www/Problems/mastsif.shtml
\end{verbatim}
ou use os comandos
\begin{verbatim}
$ wget ftp://ftp.numerical.rl.ac.uk/pub/cuter/mastsif_small.tar.gz 
$ wget ftp://ftp.numerical.rl.ac.uk/pub/cuter/mastsif_large.tar.gz
\end{verbatim}
Descompacte os problemas pequenos com o comando
\begin{verbatim}
$ tar -zxvf mastsif_small.tar.gz 
\end{verbatim}
Se quiser, descompacte os problemas grandes também. Note que estes problemas são
muito mais pesados que os pequenos, então aconselho baixá-los apenas se for necessário.
\begin{verbatim}
$ tar -zxvf mastsif_large.tar.gz
\end{verbatim}
Iremos começar a instalar o CUTEr. Para isso, vamos instalar o decodificador
dos problemas SIF (que faz parte da biblioteca CUTEr).
Entre na pasta \verb+sifdec2+ criada em \verb+$HOME/libraries/CUTEr+.
\begin{verbatim}
$ cd $HOME/libraries/CUTEr/sifdec2
\end{verbatim}
Agora digite o comando
\begin{verbatim}
$ ./install_sifdec
\end{verbatim}
A instalação do sifdec irá pedir que você escolha
\begin{itemize}
 \item Plataforma (no nosso caso será 5 - PC)
 \item Sistema Operacional (no nosso caso será 2 - linux)
 \item Compilador Fortran (no nosso caso será 7 - GNU gfortran)
 \item Precisão (no nosso caso será D - double)
 \item Tamanho (no nosso caso será L - large, porém pode ser necessário mudar para C - custom,
dependendo do software e dos testes que você utilizará. Nesse caso edite o arquivo na pasta
\verb+build/arch/size.custom+ antes de continuar.)
\end{itemize}
O instalador vai lançar a mensagem
\begin{verbatim}
 By default, SifDec with your selections will be installed in
  /home/abel/libraries/CUTEr/sifdec2/SifDec.large.pc.lnx.gfo
 Is this OK (Y/n)?
\end{verbatim}
Anote o nome do diretório que será criado e aperte enter. No nosso caso, será \\
 \verb+SifDec.large.pc.lnx.gfo+. Para a próxima mensagem, escreva `n' e aperte enter.

Agora vamos acrescentar as variáveis de sistema para que o instalador e os softwares
que formos utilizar saibam onde está o CUTEr. Edite o arquivo \verb+$HOME/.bashrc+ e
adicione as seguintes linhas
\begin{verbatim}
ROOTCUTER="$HOME/libraries/CUTEr"
export CUTER="$ROOTCUTER/cuter2"
export MYCUTER="$CUTER/CUTEr.large.pc.lnx.gfo" 
export SIFDEC="$ROOTCUTER/sifdec2"
export MYSIFDEC="$SIFDEC/SifDec.large.pc.lnx.gfo" 
export MASTSIF="$ROOTCUTER/mastsif" 
export MANPATH="$CUTER/common/man:$SIFDEC/common/man:$MANPATH" 
export PATH="$MYCUTER/bin:$MYSIFDEC/bin:$PATH"
\end{verbatim}
Se for necessário, mude a variável \verb+ROOTCUTER+, e o diretório que eu mandei anotar. 
Note que já criamos um diretório para o CUTEr, supondo que ele siga as mesmas opções
que o SifDec. 
Para que essas mudanças façam efeito é necessário usar o comando
\begin{verbatim}
$ source $HOME/.bashrc
\end{verbatim}
ou reiniciar o terminal.
Agora entre na pasta que criamos anteriormente. Para isso basta fazer
\begin{verbatim}
$ cd $MYSIFDEC
\end{verbatim}
Se deu algum erro, você pode ter errado o nome na definição da variável. Verifique que a
pasta realmente existe e refaça os passos se necessário.

Agora rode o comando
\begin{verbatim}
$ ./install_mysifdec
\end{verbatim}
Ao final desse comando, a frase
\begin{verbatim}
install_mysifdec : Do you want to 'make all' in
 /home/abel/libraries/CUTEr/sifdec2/SifDec.large.pc.lnx.gfo now (Y/n)?
\end{verbatim}
aparecerá. Aperte enter. Agora vamos instalar o CUTEr. Para isso, faça
\begin{verbatim}
$ cd $CUTER
$ ./install_cuter
\end{verbatim}
A instalação do cuter irá pedir que você escolha
\begin{itemize}
 \item Plataforma (no nosso caso será 5 - PC)
 \item Sistema Operacional (no nosso caso será 2 - linux)
 \item Compilador Fortran (no nosso caso será 7 - GNU gfortran)
 \item Compilador C (no nosso caso será 4 - g++)
 \item Precisão (no nosso caso será D - double)
 \item Tamanho (no nosso caso será L - large, com a mesma observação sobre o tamanho.)
\end{itemize}
Quando necessário, aperte enter para confirmar a continuação do programa. Se tudo deu
certo, o CUTEr deve estar instalado. Caso tenha acontecido algum erro, reveja as variáveis
e volte os passos. Pode ser necessário apagar tudo que você baixou e baixar novamente (em 
último caso). Para verificar que realmente está tudo funcionando, faça
\begin{verbatim}
$ mkdir -p $HOME/libraries/CUTEr/Testing
$ cd $HOME/libraries/CUTEr/Testing
$ runcuter -p gen -D ROSENBR
\end{verbatim}
Se nenhum erro aparecer, e aparecer várias informações do problema, então o
CUTEr foi instalado corretamente. Agora vamos criar algumas interfaces de exemplo.


Referências

[1] http://cuter.rl.ac.uk/cuter-www 
[2] https://magi-trac-svn.mathappl.polymtl.ca/Trac/cuter 
[3] http://www.math.utep.edu/Student/rsanchez/files/CUTErManual.pdf

\end{document}