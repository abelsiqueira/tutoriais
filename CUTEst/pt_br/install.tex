%!TEX root = cutest.tex
\section{Instalação}
O CUTEst é um ambiente de testes para métodos de otimização. 
Ele foi feito em Fortran, com suporte a C. Para usar em C++, o trabalho é um
pouco maior. Depois de entender o processo descrito aqui para C, verifique o
trabalho de \cite{bib:abel-dcicpp} como exemplo.
Para instalar o CUTEr você irá precisar dos programas:
\begin{itemize}
 \item gawk,
 \item gcc (ou outro compilador de C),
 \item gfortran (ou outro compilador de fortran),
 \item svn (subversion),
 \item algum editor de texto (gedit, vim, etc.).
\end{itemize}
Com os pacotes instalados, iremos criar uma pasta para o CUTEst. Você pode criar
uma pasta separada no sistema ou instalar tudo no seu diretório principal. No
meu caso, irei criar uma pasta chamada \verb+libraries+ na minha pasta pessoal,
e criar uma pasta para o CUTEst dentro dessa pasta. Ajuste os comandos de acordo
com sua escolha.

\subsection{Preparação e download}
O CUTEst trabalha com quatro diretórios: archdefs, cutest, sifdecode e sif.
Você precisa definir uma variável de sistema para cada um desses diretórios.
Vamos fazer isso logo.

Inclua no seu arquivo \verb+$HOME/.bashrc+ as seguintes linhas:
\begin{code}{bash}
LIBSDIR="$HOME/libraries"
export ARCHDEFS="$LIBSDIR/archdefs"
export SIFDECODE="$LIBSDIR/sifdecode"
export CUTEST="$LIBSDIR/cutest"
export MASTSIF="$LIBSDIR/sif"
\end{code}
Feche o terminal e abra um novo, e vamos começar a instalação.

Digite
\begin{terminal}
$ mkdir -p $LIBSDIR
$ cd $LIBSDIR
\end{terminal}
Agora você precisa fazer o download das bibliotecas usando o subversion. Para
detalhes veja \cite{bib:cutest}. Sem detalhes, copie o arquivo abaixo para um
arquivo \verb+downcutest.sh+ e salve-o na pasta \verb+$LIBSDIR+.
\begin{code}{bash}
#!/bin/bash
cmd="svn checkout -q --username anonymous"
url="http://ccpforge.cse.rl.ac.uk/svn/cutest/"
for name in archdefs sifdecode cutest sif
do
  echo "Downloading $name"
  $cmd $url/$name/trunk ./$name
done
\end{code}
Faça esse arquivo ficar executável e execute-o com os comandos
\begin{terminal}
$ chmod a+x downcutest.sh
$ ./downcuter.sh
\end{terminal}
Caso não funcione alguma coisa, você pode retirar o \verb+'-q'+ da chamada do
svn para verificar qual o erro.

\subsection{Instalação}
Para instalar o CUTEst, comece entrando no diretório do archdefs
\begin{terminal}
$ cd $ARCHDEFS
\end{terminal}
Daí, execute o script \verb+install_optsuite+ e siga as instruções
\begin{terminal}
$ ./install_optsuite
\end{terminal}
As perguntas feitas são:
\begin{terminal}
Do you wish to install GALAHAD (Y/n)?
Do you wish to install CUTEst (Y/n)?
Do you require the CUTEst-Matlab interface (y/N)?
Select platform
Select operating system
Would you like to review and modify the system commands (y/N)?
Select fortran compiler
Would you like to review and modify the fortran compiler settings (y/N)?
Select C compiler
Would you like to review and modify the C compiler settings (y/N)?
Would you like to compile SIFDecode ... (Y/n)?
Would you like to compile CUTEst ... (Y/n)?
Which precision do you require for the installed subset (D/s/b) ?
\end{terminal}
As etapas acima são totalmente dependentes da sua configuração. A maior parte
vem com o default que você provavelmente gostará de escolher. No caso, o
GALAHAD eu não selecionei, e as outras opções foram default. A plataforma que
eu escolhi foi um PC com processador de 64 bits genérico. O sistema operacional
é linux. Os compiladores de fortran e C são o gfotran e o gcc, respectivamente.

Ao fim da instalação, o script imprime uma linha tipo
\begin{terminal}
export MYARCH="xxxxxxxxxxx"
\end{terminal}
Copie essa linha para o seu \verb+$HOME/.bashrc+ e reinicie o terminal. 
No meu caso, de acordo com
minhas escolhas para o instalar, seu MYARCH é \verb+pc64.lnx.gfo+.
Se nenhum erro apareceu, você deve ter o CUTEst instalado.

\subsection{Testando o CUTEst}

Para testar, vamos criar um diretório temporário e rodar o CUTEst com pacotes
genéricos, feitos para teste.
\begin{terminal}
$ cd $(mktemp -d)
$ runcutest -p genc -D ROSENBR
$ runcutest -p gen77 -D ROSENBR
$ runcutest -p gen90 -D ROSENBR
\end{terminal}
Cada comando destes 3 últimos deve retornar uma tela com estatísticas CUTEst.
Eles são, respectivamente, para C, fortran 77 e fortran 90.
