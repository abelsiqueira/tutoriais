%!TEX root=cutest.tex
\subsection{Exemplos em C}
Fizemos três implementações do método de máxima descida. A primeira é uma implementação
que não leva me conta o CUTEr, e depois adapta o CUTEr para o problema. A segunda já leva
em conta o formato das funções do CUTEr e faz pouca adaptação posteriormente. A terceira
usa exatamente as funções do CUTEr, não necessitando de adaptação.

Cada implementação do método de máxima descida consiste de dois arquivos:
\verb+steepest_descent.h+ e \verb+steepest_descent.c+. No .h, definimos que funções
iremos chamar, e uma estrutura com as informações da execução. As funções para a 
primeira implementação são
\begin{itemize}
 \item \verb+double Norm (double * x, unsigned int n);+ \\
Esta função calcula a norma 2 de um vetor \verb+x+ com tamanho \verb+n+.
 \item \verb+double NormSqr (double * x, unsigned int n);+ \\
Esta função calcula o quadrado da norma 2 de um vetor \verb+x+ com tamanho \verb+n+. É mais
rápido que a função \verb+Norm+ pois não envolve raiz quadrada.
 \item \verb+SteepestDescent (double * x, unsigned int n, Status * status);+ \\
Esta é a função que encontra o ponto estacionário. \verb+x+ entra como ponto inicial e 
sai como a solução. \verb+n+ é a dimensão do problema e \verb+status+ é um ponteiro para
a estrutura de informações.
 \item \verb+SD_Print (double * x, unsigned int n, Status * status);+ \\
Esta função imprime o vetor \verb+x+ e as informações da execução do problema.
\end{itemize}
A estrutura do problema
\begin{code}{C}
typedef struct _Status {
  unsigned int iter;
  double f, ng;
  unsigned int n_objfun, n_gradfun;
} Status;
\end{code}
\verb+iter+ é o número de iterações que o algoritmo executou, \verb+f+ é o valor da
função objetivo na solução, \verb+ng+ é a norma do gradiente da função objetivo,
\verb+n_objfun+ é o número de cálculos da função objetivo e \verb+n_gradfun+ é o número
de cálculos do gradiente. Mostraremos as diferenças das outras implementações posteriormente.

Nosso arquivo .c contém as definições das funções acima, e contém uma declaração de função
usada para acessar a função objetivo e o gradiente. Na primeira implementação, essa declaração
é
\begin{code}{C}
double objfun  (double * x, unsigned int n);
void   gradfun (double * x, unsigned int n, double * g);
\end{code}
As funções \verb+Norm+, \verb+NormSqr+ e \verb+SD_print+ são idênticas para todas as
implementações e serão deixadas de fora. A implementação do método em si encontra-se abaixo.
\scriptsize
\lstinputlisting[style=codestyle, language=C]
{../Examples/CExample1/steepest_descent.c}
\normalsize
Um código de exemplo para esse teste é
\scriptsize
\lstinputlisting[style=codestyle, language=C]
{../Examples/CExample1/test1.c}
\normalsize
Este código implementa o problema de minimizar $f(x) = \meio\norma{x}^2$ em duas dimensões.
Note como temos que declarar as funções \verb+objfun+ e \verb+gradfun+. Sem elas teríamos
erros na compilação. Veja os arquivos \verb+test2.c+ e \verb+test3.c+
para outros exemplos.

A interface para o CUTEr é o arquivo \verb+c_example1main.c+:
\scriptsize
\lstinputlisting[style=codestyle, language=C]
{../Examples/CExample1/c_example1main.c}
\normalsize
Note que também é necessário definir as funções \verb+objfun+ e \verb+gradfun+.
Essas funções, por sua vez, chamam as funções correspondentes em CUTEr para esse
serviço. A função \verb+UFN+ calcula o valor da função objetivo e a função
\verb+UGR+ calcula o valor do gradiente. Note que quando compilamos uma função
em Fortran, ele recebe um nome em letras minúsculas e com um \_ (underline) na
frente (no caso do gfortran. Outros compiladores podem divergir). O arquivo
\verb+cuter.h+ define macros para todas as funções do CUTEr serem chamados com
letras maiúsculas. Note ainda que a função \verb+UFN+ recebe um ponteiro para
\verb+int+ e dois ponteiros para \verb+double+, sendo o primeiro para o vetor
$x$ e o segundo para o valor da função objetivo. As funções do CUTEr para C
recebem ponteiros em todos os valores. Os valores que são vetores não precisam
de um ponteiro adicional. Note também que como utilizamos \verb+unsigned int+
para os tamanhos, tivemos que converter os valores para \verb+int+.

Veja agora o Exemplo 2. Nesse exemplo consideramos que as funções devem estar no mesmo
formato que a função do CUTEr.
\begin{code}{C}
void ufn (int * n, double * x, double * f);
void uofg (int * n, double * x, double * f, double * g, long int * grad);
\end{code}
O nome das funções foram escolhidas para seguir exatamente o formato do CUTEr, mas aqui
elas poderiam ser qualquer coisa. Note que a função \verb+uofg+ foi utilizada no lugar
da função \verb+ugr+. Essa função já calcula a função objetivo e o gradiente, sendo mais
rápida que chamadas individuais. Com essa mudança, esse programa é levemente mais rápido
que o outro. O resto do arquivo foi mudado de acordo a seguir essas mudanças.

A interface também teve uma mudança na definição das funções:
\begin{code}{C}
void ufn (int * n, double * x, double * f) {
  UFN(n, x, f);
}

void uofg (int * n, double * x, double * f, double * g, long int * grad) {
  UOFG(n, x, f, g, grad);
}
\end{code}
Essa interface fica muito mais natural de ser utilizada num problema com CUTEr. Não precisamos
converter nenhuma variável, simplesmente fazer a chamada da função com os parâmetros já dados.
Note, no entanto, que estamos utilizando \verb+long int+ para as variáveis lógicas do CUTEr.
Essa é uma definição do arquivo \verb+cuter.h+. Se mudarmos essa definição, então devemos
criar uma variável \verb+logical GRAD = *grad+ e chamar a função com
\verb+UOFG(n, x, f, g, &GRAD)+.

A última interface já declara as funções que o CUTEr irá definir. Então no arquivo .c temos
\begin{code}{C}
void ufn_ (int * n, double * x, double * f);
void uofg_ (int * n, double * x, double * f, double * g, long int * grad);
\end{code}
e no arquivo da interface não temos nenhuma definição de função. Lembre-se que o Fortran
compilado com gfortran cria as funções com minúsculas e \_ na frente. Se mudarmos o
compilador pode não funcionar. 
Além disso, a definição dessas funções é, na verdade
\begin{code}{Fortran}
void UFN( integer *n, doublereal *x, doublereal *f );
void UOFG( integer *n, doublereal *x, doublereal *f, doublereal *g,
      logical *grad);
\end{code}
e esses tipos são definidos no arquivo \verb+cuter.h+, podendo ser alterados pelo usuário.
Se isso acontecer, é necessário mudar todo o programa.

Uma alternativa é utilizar \verb+typedef+s para definir os tipos próprios, e deixar o
acesso desses tipos para o usuário (assim como o arquivo \verb+cuter.h+). Dessa maneira,
se o usuário tiver necessidade de mudar o arquivo \verb+cuter.h+, ele também pode (deve)
mudar o arquivo com esses \verb+typedef+s.

Para compilar a interface CUTEr dos testes utilize
\begin{terminal}
$ make cuter
\end{terminal}
Para rodar os testes utilize o comando
\begin{terminal}
$ runcuter -p c_example# -D BARD
\end{terminal}
onde \# é o número do exemplo e BARD é um dos problemas em que esse exemplo converge. Note
que é preferível criar uma pasta separada para rodar os testes, já que eles geram lixo na
pasta.
